\documentclass[onecolumn, 12pt]{article}

\usepackage[latin1]{inputenc}   
\usepackage{amsmath}
\usepackage{algorithm}
\usepackage{algorithmic}
\usepackage{listings}
\usepackage{hyperref}
%\usepackage[T1]{fontenc}

%\usepackage[francais]{babel}     
 %\usepackage{layout}    
 %\usepackage[top=2cm, bottom=2cm, left=2cm, right=2cm]{geometry} 
 %\usepackage{setspace}
 %\usepackage{soul}
 %\usepackage{color} 
  %\usepackage{verbatim}
  %\usepackage{moreverb}
  %\usepackage{listings}
 %\usepackage{url}
\usepackage{float}
 \usepackage{graphicx}
 \usepackage[outdir=./]{epstopdf}
%\usepackage{caption}
 %\usepackage{setspace}
 
% \setstretch{1.0}
 
 %\makeatletter
 
\graphicspath{
	{./images/}
}

 %\def \@maketitle{   % custom maketitle 
%{\Large \bfseries \color{black} \@title}
%hrule permet de dessiner une ligne droite sous le titre
%{\scshape Student:} \@author ~ at  \@date \par \smallskip \hrule \bigskip 
%}
%can contain definition of particular /section
%\makeatother
 
 \title{Test entretien Pradeo}
 \author{Wilfried Ehounou}
 %\date{01/06/15}
 
 \begin{document}
%\chapter{Modle de donnes}
\maketitle
\tableofcontents
%\listoffigures

% definition macro
\def\R{\mbox{I\hspace{-.15em}R}}
\def\N{\mbox{I\hspace{-.15em}N}}
\def\Z{\mbox{I\hspace{-.15em}Z}}
\def\Q{\mbox{I\hspace{-.15em}Q}}

\section{Explication of the software: lookupMalware}

The goal of the {\em lookupMalware} application is to analyse a json file containing the characteristics of many applications (i.e $100$) and return a response talking about  a malware inside each application.
\newline
{\em lookupMalware} is a web-application developped with Flask and Python. 
The application sends an API request via {\em virusTotal} platform and has got responses from  a lot of antivirus. Each antivirus detects a malware with the True answer and in the opposite case, the answer is False. We calculate the percent of False answers (named {\em percent\_ans}) and consider that one like a probability of having any malwares into one scanned application. For example, $percent\_ans = 1$ means there are no malwares into application  while there exists a malware when $percent\_ans = 0$. The  {\em percent\_ans} values  are categorized into $4$ labels :
\begin{itemize}
	\item {\em mauvais} : $percent\_ans \in [0, 0.5[$ means that the risk of having a malware is very high. A malware infected the scanned application. 
	\item {\em moyen} :  $percent\_ans \in [0.5, 0.7[$ means thats the risk of having a malware is high. The scanned application most likely has a malware.
	\item {\em Acceptable} : $percent\_ans \in [0.7, 0.9[$ means thats the risk of having a malware is low. The scanned application most likely has not a malware.
	\item {\em Excellent} : $percent\_ans \in [0.9, 1]$ means thats there are no malwares into the  the scanned application. 
\end{itemize}
The responses of API requests  are summarized in a dataframe and are shown in the HTML table. 
In the table, the rows denote the scanned application names while the columns describe some informations about API resquests and the labels assigned of each scanned application. 
The column names are :
	\begin{itemize}
		\item percent\_ans : the percent of False responses, value between $0$ and $1$.
		\item percent\_ans\_lab : the label assigned at each percent value.
		\item Scan\_id : identity of the virusTotal scan. {\em None} Scan\_id means server does not analyse application.
		\item Status\_code : status of the request. The code $200$ denotes  the request was correctly handled by the server and no errors were produced. When server returns anything, the code will be $203$, $403$ or $400$. The code {\em nan} implies bad SHA1 application.
		\item Response\_code : the code of the response. If response\_code $=0$ then the application is not in the virusTotal database. The Response\_code will be $-2$ if the application is still queued for analysis. The Response\_code will be $None$ if the Status\_code is $203$, $403$ or $400$. If the response\_code $=1$ then we could retrieve the item you searched.
		\item Date : the date in the application description.
		\item Version : the application version in the description file.
 		\item Commentaire : it describes how well the request was done. The string "OK" means processing was done without errors.
	\end{itemize}
The web application {\em virusTotal} has some limitations about the created account. When you have not a premium account ({\bf Like me}), then the API response is limited at $4/min$ and $1000$ resquests in one day. %In the other way, the number of resquests are more 
With theses contraints, we defined $2$ parts in {\em lookupMalware} application on the main web page (index.html):
\begin{itemize}
	\item The first part is the click on {\em Click here to analyse files without considering virusTotal API's request limitations}. We process all the applications one time, summarized all API responses in a dataframe and show them  on the HTML table.
	\item The second part is the click on {\em Click here to analyse files with respecting virusTotal API's quota}. We consider API response quotas (named {\em QUOTA=4}) and process applications by chunk. Each chunk bloc has  $QUOTA$ applications and the API responses of this bloc are shown in the HTML table. To go through next chunk bloc, we have to wait {\em PUBLIC\_API\_SLEEP\_TIME} seconds (may be defined on command line execution).
\end{itemize}

% how to execute this web application 
\subsection{Execution of {\em lookupMalware}}
This application is coded on Ubuntu OS with Python, Flask and somes Python libraries like Pandas. In order to run this application, i recommend you to install Anaconda distribution because it is a best tool for data science project and it has its own GUI environment management tool with its navigator. 
\newline 
To install Anaconda distribution, follow the steps in the link below:
\newline
\href{https://www.datacamp.com/community/tutorials/installing-anaconda-windows}{https://www.datacamp.com/community/tutorials/installing-anaconda-windows}
\newline

Create an conda environment like that 
\begin{lstlisting}[language=bash]
$ conda create --name pradeo python=3.7 spyder=3.3.6 
pandas=0.24.2 flask=1.0.2 numpy=1.17.4 jupyter=1.0.0
\end{lstlisting}

Activate this environment

\begin{lstlisting}[language=bash]
$ source activate pradeo
\end{lstlisting}

Run the lookupMalware application like that
\begin{lstlisting}[language=bash]
$ python3 lookup_malware.py 30
\end{lstlisting}

the value $30$ denotes the time to wait between $2$ resquests to virusTotal server.
\newline

Open the firefox browser and write :
\begin{lstlisting}[language=bash]
http://127.0.0.1:5000/
\end{lstlisting}




% some image on the web application
\subsection{Views of  {\em lookupMalware} }

The index of {\em  lookupMalware} looks like:
	\begin{figure}[H]
		\includegraphics[scale=0.5400]{image_index.eps}
		\caption{The index of  lookupMalware app: 1) the choice $1$ leads to the webpage without resquest limitations. 2) the choice $2$ leads to the webpage with resquest quotas.}
		\label{fig:imageIndex}
	\end{figure}

On figure \ref{fig:imageIndex}, click on $1$ for going to the webpage without resquest limitations and $2$ otherwise.
 
\subsubsection{ Execution without resquest limitations}
\label{sectionLimitation}
To analyse a json file of various applications, you have to click on button {\em browse} to import  this file. You click on {\em submit} buttom to launch analysis.
The result is shown on the section {\em Result of file analysis} with the file name alongside.
\newline
On figure \ref{fig:imageLimitation}, we note that a lot of rows on the table have a red label in the {\em percent\_ans\_lab} column. It means that resquests on the server have not processed because of limitations. That is why the values of the {\em Scan\_id} columns are {\em None} and {\em Status\_code} columns have got $204$ code. 

	\begin{figure}[H]
		\includegraphics[scale=0.5900]{image_without_quota_select_file.eps}
		\caption{Request limitation webpage with a view of analysis table.}
		\label{fig:imageLimitation}
	\end{figure}

\subsubsection{ Execution with resquest limitations}
As it explained on the previous section \ref{sectionLimitation}, the result in the webpage is shown on many tables overlayed on each other. Each table contains $4$ scanned applications, $4$ denotes the number of server response in $1$ min.
On figure \ref{fig:imageQuota}, we have $2$ HTML tables with the same headers.  There are no rows with the $204$ code. But one cell has the {\em mauvais} label because the server response has not {\em scan\_id} key.
	\begin{figure}[H]
		\includegraphics[scale=0.4400]{image_with_quota_select_file.eps}
		\caption{A view of formatted server responses in 2 tables (in red).}
		\label{fig:imageQuota}
	\end{figure}


\section{Defi 1: Analyse de flux sortants}

Let's assume we have installed Windows $10$, CNEWS application and we could login on Windows $10$.
\newline
In order to find out the servers to which CNEWS application connects, we capture and analyze HTTP traffic with {\em Wireshark} on the default ethernet network interface ($eth0$). 
\newline
Before starting, we must run CNEWS application. This installed version is very old (i.e $2.2.1.0$) and we can't update because the installed windows 10 on virtual machine is an {\em N edition} without upgrading OS and we have not a valid key.
\newline 
The first step on {\em Wireshark} is to activate the interface $eth0$ and launch the scan like the figure \ref{fig:wiresharkScan}.
	\begin{figure}
		\includegraphics[scale=0.4400]{image_wireshark_scan.eps}
		\caption{Scan the eth0 interface: 1) launch the scan 2) stop the scan.}
		\label{fig:wiresharkScan}
	\end{figure}
We filter the TCP and HTTP traffics because the $3$ packets on TCP in the figure \ref{fig:wiresharkTCPHTTP} initialize the connection of the server and HTTP packet send {\em GET} resquest to the server. We write in the filter box 
\begin{lstlisting}[language=bash]
> tcp || http
\end{lstlisting}
	 \begin{figure}[H]
		\includegraphics[scale=0.5400]{image_wireshark_http_tcp_traffic.eps}
		\caption{Selection of TCP and HTTP traffics: 1) the filter box, 2) the server connection, 3) the GET request.}
		\label{fig:wiresharkTCPHTTP}
	\end{figure}

{\em Wireshark} has $3$ windows. The top window shows the captured packet list and the middle window observes the packet details. The bottom window shows the current packet in a hexdump style.\newline
In the figure \ref{fig:wiresharkHttpHost}, we select the HTML packet in the top window, expand Hypertext Transfer Protocol and observe the {\em GET} request in the middle Wireshark packet details. The field to identify the URL server address is {\em Host}. The URL of the server is {\em service.canal-plus.com}
	\begin{figure}[H]
		\includegraphics[scale=0.4400]{image_wireshark_htmlPacket_host.eps}
		\caption{Discovery the URL server: 1) select the HTTP packet, 2) open the Hypertext Transfer Protocol trame, 3) show the URL server}
		\label{fig:wiresharkHttpHost}
	\end{figure}

To retrieve CNEWS server connection list, we have to click on a section of the application and analyze the TCP and HTTP packets as explained above. The list of CNEWS servers, i have been able to collect, is:
\begin{itemize}
	\item {\em service.canal-plus.com}
	\item {\em hls-m006.live-lv3.canalplus-cdn.net}
	\item {\em stat.canal-plus.com}
	\item {\em us-cplus-aka.canal-plus.com}
	\item {\em service.itele.fr}
\end{itemize}


\end{document}